\chapter{Estado del arte}
\label{cap:capitulo2}

\begin{flushright}
\begin{minipage}[]{10cm}
\emph{No hay nada como el sueño para crear el futuro.}\\
\end{minipage}\\

Victor Hugo, Los miserables\\
\end{flushright}

\vspace{1cm}

En este capítulo se presentan algunas de las investigaciones y trabajos más destacados en el desarrollo de robots asistenciales, especialmente aquellos centrados en el cuidado de personas mayores mediante el uso de IA.

\vspace{0.5cm}

El aumento de la población envejecida, de personas con capacidades reducidas y la escasez de cuidadores han generado la necesidad de desarrollar soluciones tecnológicas orientadas a mejorar la calidad de vida de estos colectivos. Tal como se recoge en \cite{s21062212}, las discapacidades como la reducción de movilidad pueden provocar una pérdida significativa de la autonomía personal. En este contexto, la incorporación de dispositivos con los que el paciente pueda interactuar y que pueda controlar de manera autónoma contribuye a reforzar su independencia y bienestar.

\vspace{0.5cm}


Por otro lado, en \cite{KARAMI2024105409} se aborda también la complejidad asociada al cuidado de pacientes con enfermedades neurodegenerativas, como el Alzheimer. Estos pacientes requieren cuidados especializados y continuos, por lo que los robots asistenciales constituyen una herramienta de gran utilidad para apoyar tanto a los pacientes como a sus cuidadores.

\vspace{0.5cm}


Asimismo, en \cite{designs6060125} se destaca la dificultad de encontrar soluciones rentables que permitan suplir la falta de profesionales cualificados, así como los beneficios potenciales del uso de robots asistenciales. Además, se subraya la importancia de adoptar un enfoque ético, centrado en la persona, que permita abordar de manera fiable sus necesidades sociales, emocionales y físicas, respetando su dignidad y privacidad. En esta línea, el diseño y el uso de los robots asistenciales sociales(SAR) deben orientarse a apoyar a los cuidadores como herramientas complementarias, en lugar de sustituir su labor \cite{integrative}.


\vspace{0.5cm}




