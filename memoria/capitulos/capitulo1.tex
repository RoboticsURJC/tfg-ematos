\chapter{Introducción}
\label{cap:capitulo1}
\setcounter{page}{1}

\begin{flushright}
\begin{minipage}[]{10cm}
\emph{ La mayor aventura es la que nos espera. Hoy y mañana aún no se han dicho. Las posibilidades, los cambios son todos vuestros por hacer. El molde de su vida en sus manos está para romper.}\\
\end{minipage}\\

J.R.R. Tolkien, \textit{El Hobbit}\\
\end{flushright}

\vspace{1cm}


La robótica ha ido evolucionando a pasos agigantados en la última mitad del siglo XX; sin embargo, ya apuntaba maneras desde mucho antes. 
En la actulidad, la integración de la inteligenia artificial y el aprendizaje automático está impulsando la robótica hacia nuevas posibilidades.

\vspace{0.5cm}

Los robots han pasado de ser herramientas especializadas con capacidades limitadas a sistemas versátiles, adapatables y con un alto grado de autonomía.
La irrupción de modelos avanzados de IA está potenciando el desarrollo de una auténtica revolución.

\vspace{0.5cm}

En el siguiente capítulo se presenta la definicón de robot,  los distintos tipos de robots existentes y algunas de sus principales aplicaciones en la sociedad con el objetivo comprender las bases de este Trabajo de Fin de Grado.\\

\section{La robótica}
\label{sec:miseccion} % etiqueta para luego referenciar esta sección

La robótica es un campo multidisciplinar que surge de la intersección de la ciencia, la ingeniería y la tecnología. Se dedica al estudio, diseño, construcción e implementación de robots. 

Pero, ¿qué es un robot? 

\vspace{0.5cm}

Un robot es una máquina programable capaz de realizar tareas de forma autónoma o semiautónoma.

\vspace{0.5cm}

Sus orígenes pueden remontarse a las antiguas civilicaciones del Antiguo Egipto, que ya desarrollaban modelos matemáticos muy avanzados y construyeron automatismos muy sofisticados, como el reloj de agua. Siglos más tarde, Leonardo da Vinci diseñó el \textit{Automa Cavaliere}~\ref{fig:Da-Vinci} , considerado uno de los primeros robots humanoides de la historia. 

\vspace{0.2cm}

\begin{figure} [h!]
  \begin{center}
    \includegraphics[width=8cm]{figs/Da-Vinci}
  \end{center}
  \caption{\textit{Automa Cavaliere} de Leonardo Da Vinci}
  \label{fig:Da-Vinci}
\end{figure}\


\vspace{0.5cm}

No obstante, no fue hasta 1921 donde se acuñó por primera vez el concepto de "robot", en la obra de  teatro llamada  “R.U.R: ROBOTS UNIVERSALES ROSSUM”  del dramaturgo checo Karel Capek. Con el paso del tiempo, este término acabaría dando nombre a toda una ingeniería: la robótica.

\vspace{0.5cm}

El siglo XX ha supuesto un antes y después en la robótica: en apenas unas décadas, se pasó  de mecanismos que realizaban tareas repetitivas a sistemas complejos y autónomos. Gracias al desarrollo de la robótica moderna, el escritor de ciencia ficción Isaac Asimov publicó \textit{Yo Robot } obra en la que aparecen las conocidas \textit{leyes de la robótica}, las cuales limitan el funcionamiento y aplicaciones de los robot para evitar que supongan una amenaza para la humanidad.

\vspace{0.5cm}

En el año 1961 la compañía estadounidense  \textit{Unimation} desarrolló el primer robot industrial programable, marcando un hito en la automatozación industrial.
A apartir de ese momento aparecieron nuevos robots en otros sectores, permitiendo automatizar ámbitos y tareas que hasta entonces eran inimaginables.

 




En los textos puedes poner palabras en \textit{cursiva}, para aquellas expresiones en sentido \textit{figurado}, palabras como \textit{robota}, que está fuera del diccionario castellano, o bien para resaltar palabras de una colección: \textit{(a)} es la primera letra del abecedario, \textit{(b)} es la segunda, etc.\\

Al poner las dos líneas del anterior párrafo, este aparecerá separado del anterior. Si no las pongo, los párrafos aparecerán pegados. Sigue el criterio que consideres más oportuno.



\subsection{Robótica industrial}
\label{sec:subseccion}

La robótica industrial es la rama de la ingería que se encarga del diseño, desarrollo y fabricación de robots que automatizan tareas repetitivas y/o peligrosas en el ámbito industrial.
Se utilizan en entornos controlados para ejecutar tareas de manera precisa y más rápida, de esta forma se mejora la eficiencia y productividad de los recursos y se reducen los errores en la cadena de produción.

\vspace{0.5cm}

El origen de los robots industriales comienza con \textit{Unimate}\ref{fig:Unimate}, patentado por George Devol en 1961.

\vspace{0.2cm}

\begin{figure} [h!]
  \begin{center}
    \includegraphics[width=8cm]{figs/Unimate}
  \end{center}
  \caption{\textit{Unimate} de George Devol}
  \label{fig:Unimate}
\end{figure}\

\vspace{0.5cm}

El desarrollo de \textit{Unimate} impulsó el desarrollo de los robots industriales permitiendo la optimización de procesos.

\vspace{0.5cm}

Hoy en día gracias a las nuevas a las nuevas tecnologías como la inteligencia artificial, el internet de las cosas o el Machine Learning se ha conseguido que los robots insdustriales sean capaces tomar decisiones en tiempo real y actuar en consecuencia.


\subsection{Robótica de Servicio}
\label{sec:subseccion}

Los robots de servicio son uno de los tipos de robots que más podemos ver en la sociedad actual.

Pero, ¿cómo se define un robot de servicio?


\subsection{Robótica Social}
\label{sec:subseccion}


\section{Vision Artificial}
\label{sec:segundaseccion}

\section{Machine Learning}
\label{sec:segundaseccion}


\section{Deep Learning}
\label{sec:segundaseccion}



No olvides incluir imágenes y referenciarlas, como la Figura \ref{fig:roomba}.

\begin{figure} [h!]
  \begin{center}
    \includegraphics[width=8cm]{figs/roomba}
  \end{center}
  \caption{Robot aspirador Roomba de iRobot.}
  \label{fig:roomba}
\end{figure}\

Ni tampoco olvides de poner las URLs como notas al pie. Por ejemplo, si hablo de la Robocup\footnote{\url{http://www.robocup.org}}.



\subsection{Números}
\label{sec:subseccion}



En lugar de tener secciones interminables, como la Sección \ref{sec:miseccion}, divídelas en subsecciones.

Para hablar de números, mételos en el entorno \textit{math} de \LaTeX, por ejemplo, $1.5Kg$. También puedes usar el símbolo del Euro como aquí: 1.500\euro.

\subsection{Listas}

Cuando describas una colección, usa \texttt{itemize} para ítems o \texttt{enumerate} para enumerados. Por ejemplo:

\begin{itemize}
 \item \textit{Entorno de simulación.} Hemos usado dos entornos de simulación: uno en 3D y otro en 2D.
 \item \textit{Entornos reales.} Dentro del campus, hemos realizado experimentos en Biblioteca y en el edificio de Gestión.
\end{itemize}\

\begin{enumerate}
 \item Primer elemento de la colección.
 \item Segundo elemento de la colección.
\end{enumerate}\

\paragraph{Referencias bibliográficas}
\label{sec:referencias}

Cita, sobre todo en este capítulo, referencias bibliográficas que respalden tu argumento. Para citarlas basta con poner la instrucción \verb|\cite| con el identificador de la cita. Por ejemplo: libros como \cite{vega12e}, artículos como \cite{vega19b}, URLs como \cite{vega19a}, tesis como \cite{vega18b}, congresos como \cite{vega18a}, u otros trabajos fin de grado como \cite{vega08b}.

Las referencias, con todo su contenido, están recogidas en el fichero \texttt{bibliografia.bib}. El contenido de estas referencias está en formato \texttt{BibTex}. Este formato se puede obtener en muchas ocasiones directamente, desde plataformas como \texttt{Google Scholar} u otros repositorios de recursos científicos.

Existen numerosos estilos para reflejar una referencia bibliográfica. El estilo establecido por defecto en este documento es APA, que es uno de los estilos más comunes, pero lo puedes modificar en el archivo \texttt{memoria.tex}; concretamente, cambiando el campo \verb|apalike| a otro en la instrucción \verb|\bibliographystyle{apalike}|. 

\

\

\

Y, para terminar este capítulo, resume brevemente qué vas a contar en los siguientes.
