\chapter{Introducción}
\label{cap:capitulo1}
\setcounter{page}{1}

\begin{flushright}
\begin{minipage}[]{10cm}
\emph{Al crear, lo único difícil es empezar: una brizna de hierba no es más fácil de hacer que un roble.}\\
\end{minipage}\\

James Russell Lowell\\
\end{flushright}

\vspace{1cm}


La robótica ha ido evolucionando a pasos agigantados en la última mitad del siglo XX.
En la actualidad, la integración de la Inteligencia Artificial (IA) y el aprendizaje automático está impulsando la robótica hacia nuevas posibilidades.  

\vspace{0.5cm}

Los robots han pasado de ser herramientas especializadas con capacidades limitadas a sistemas versátiles, adaptables y con un alto grado de autonomía.
La irrupción de modelos avanzados de IA está potenciando el desarrollo de una auténtica revolución.

\vspace{0.5cm}

En la sección 1.1 se presenta la definición de robot,  los distintos tipos de robots existentes y algunas de sus principales aplicaciones en la sociedad.\\

\section{La robótica}
\label{sec:La robótica} % etiqueta para luego referenciar esta sección

La robótica es un campo multidisciplinar que surge de la unión de la ciencia, la ingeniería y la tecnología. Se dedica al estudio, diseño, construcción e implementación de robots. 
Un robot puede definirse como una máquina programable capaz de realizar tareas de forma autónoma o semi-autónoma.

\vspace{0.5cm}

Los orígenes de la robótica se remontan a las antiguas civilizaciones del Antiguo Egipto, que ya desarrollaban modelos matemáticos muy avanzados y construyeron automatismos muy sofisticados, como el reloj de agua (Figura \ref{fig:origen} a). Siglos más tarde, Leonardo da Vinci diseñó el \textit{Automa Cavaliere} (Figura \ref{fig:origen} b) , considerado uno de los primeros robots humanoides de la historia. 

\vspace{0.2cm}

  \begin{figure}[H]
    \begin{center}
      \subcapcentertrue
      \subfigure[Reloj de agua (imagen generada por IA)]{\includegraphics[width=61mm]{figs/reloj-agua}}
      \hspace{3mm}
      \subfigure[\textit{Automa Cavaliere} de Leonardo Da Vinci]{\includegraphics[width=62mm]{figs/Da-Vinci}}
    \end{center}
    \caption{Primeros sistemas robóticos conocidos}
    \label{fig:origen}
  \end{figure}


No obstante, no fue hasta 1921 donde se acuñó por primera vez el término de \textit{robot}, en la obra de  teatro llamada  \textit{R.U.R: ROBOTS UNIVERSALES ROSSUM}  del dramaturgo checo Karel Capek.
Con el paso del tiempo, este término acabaría dando nombre a toda una ingeniería: la robótica.  \cite{iniciorob}

\vspace{0.5cm}

El siglo XX ha supuesto un antes y después en la robótica: en apenas unas décadas, se pasó  de mecanismos que realizaban tareas repetitivas a sistemas complejos y autónomos. Gracias al desarrollo de la robótica moderna, el escritor de ciencia ficción Isaac Asimov publicó \textit{Yo Robot}, obra en la que aparecen las conocidas \textit{Leyes de la robótica}, las cuales limitan el funcionamiento y aplicaciones de los robot para evitar que supongan una amenaza para la humanidad.

\vspace{0.5cm}

En las siguientes secciones se describen algunos de los tipos de robots presentes en la sociedad actual, como los robots de servicio, sociales y asistenciales, entre otros.
 
\vspace{0.5cm}




\subsection{Robótica industrial}
\label{sec:Robótica industrial}

La robótica industrial es la rama de la ingeniería que se encarga del diseño, desarrollo y fabricación de robots que automatizan tareas repetitivas y/o peligrosas en el ámbito industrial. Estos robots se utilizan en entornos controlados para ejecutar tareas de manera precisa y eficiente, mejorando la productividad de los procesos y reduciendo los errores en la cadena de producción.

\vspace{0.5cm}

El origen de los robots industriales se remonta al desarrollo de \textit{Unimate}\footnote{https://unimaterobotica.com/} (Figura \ref{fig:Unimate}), patentado por George Devol en 1961.

\vspace{0.2cm}

\begin{figure} [h!]
  \begin{center}
    \includegraphics[width=8cm]{figs/Unimate}
  \end{center}
  \caption{\textit{Unimate}, de George Devol}
  \label{fig:Unimate}
\end{figure}\


La introducción de \textit{Unimate} impulsó el desarrollo de nuevos robots industriales, como los robots colaborativos desarrollados por empresas como ABB\footnote{https://new.abb.com/products/robotics/es/robots} (Figura \ref{fig:industria} a) y los robots \textit{SCARA (Selective Compliance Assembly Robot Arm)} fabricados por compañías como Yamaha\footnote{https://global.yamaha-motor.com/business/robot/lineup/ykxg/}  (Figura \ref{fig:industria} b), lo que permitió optimizar los procesos industriales.


\vspace{0.2cm}

  \begin{figure}[H]
    \begin{center}
      \subcapcentertrue
      \subfigure[YuMi® - IRB 14000]{\includegraphics[width=50mm]{figs/yumi}}
      \hspace{22mm}
      \subfigure[Orbit type YK350TW/YK500TW]{\includegraphics[width=50mm]{figs/orbit}}
    \end{center}
    \caption{Robots industriales}
    \label{fig:industria}
  \end{figure}



Además, favoreció la expansión del uso de la robótica en otros sectores, como la robótica de servicio.

\vspace{0.5cm}



\subsection{Robótica de Servicio}
\label{sec:Robótica de Servicio}

En la actualidad, los robots de servicio se han convertido en uno de los tipos de robots más habituales en la sociedad.
Pero, ¿cómo se define un robot de servicio?

\vspace{0.5cm}

Según la\textit{ Federación internacional de Robótica}\footnote{https://ifr.org/} un robot de servicio se define como\textit{ robot de uso personal o profesional que realiza tareas útiles para humanos o equipos} \cite{ISO8373}.

\vspace{0.5cm}

Las aplicaciones de los robots de servicio son muy amplias y abarcan campos muy diversos, como la agricultura, la construcción, la logística, la defensa o la medicina, entre muchos otros.

\vspace{0.5cm}

En el ámbito de la medicina destaca el robot Da Vinci (Figura \ref{fig:Robot-Medicina}), un sistema quirúrgico controlado por un cirujano, que permite mejorar las capacidades del profesional durante las intervenciones. Este sistema contribuye a aumentar la precisión, reduciendo así los temblores o movimientos involuntarios que pueda tener el cirujano. Además, el robot viene equipado con múltiples cámaras que proporcionan una visión más detallada del área a intervenir.


\vspace{0.2cm}

\begin{figure} [h!]
  \begin{center}
    \includegraphics[width=8cm]{figs/Robot-Medicina}
  \end{center}
  \caption{\textit{Robot Da Vinci}, de George Devol}
  \label{fig:Robot-Medicina}
\end{figure}\

La incorporación de estas nuevas tecnologías en el campo de la medicina suponen una gran ventaja tanto para el profesional sanitario como para el paciente, ya que el uso de este sistema permite realizar intervenciones menos invasivas, favoreciendo una mejor recuperación, con menor sangrado, cicatrices más reducidas y una disminución de las complicaciones postoperatorias.


\vspace{0.5cm}

Otra de las aplicaciones más extendidas de la robótica de servicio son los robots de limpieza. Estos robots se caracterizan por su capacidad de navegación autónoma, basada en la generación de mapas internos que les permiten desplazarse de forma eficiente y evitar obstáculos. 
Existe una amplia variedad de modelos en el mercado, desarrolladas por compañías como iRobot\footnote{https://www.irobot.es/} (Figura \ref{fig:limpieza} a) o Ecovan\footnote{https://www.ecovacs.com/es} (Figura \ref{fig:limpieza} b), que incorporan distintos niveles de tecnología y funcionalidades. 

\vspace{0.2cm}

  \begin{figure}[H]
    \begin{center}
      \subcapcentertrue
      \subfigure[Roomba-j7]{\includegraphics[width=50mm]{figs/Roomba-j7}}
      \hspace{22mm}
      \subfigure[Ecovacs Deebot OZMO N8 PRO+]{\includegraphics[width=50mm]{figs/Ecovacs}}
    \end{center}
    \caption{Robots de limpieza}
    \label{fig:limpieza}
  \end{figure}


Dentro de la robótica de servicio, la logística constituye uno de los ámbitos en los que estos sistemas han experimentado un mayor desarrollo.
Los robots logísticos se emplean en tareas como el transporte de mercancías dentro de almacenes, optimizando los procesos, reduciendo los costes logísticos y aumentando la productividad.

\vspace{0.5cm}

Existen principalmente dos tipos de robots logísticos: los robots móviles autónomos (AMR)\footnote{https://www.mecalux.es/} (Figura \ref{fig:logistica} a) y los vehículo de guiado automático (AGV)\footnote{https://agvrobotics.es/} (Figura \ref{fig:logistica} b). 

\vspace{0.2cm}

  \begin{figure}[H]
    \begin{center}
      \subcapcentertrue
      \subfigure[AMR 1500 Pallet Lifter]{\includegraphics[width=60mm]{figs/AMR}}
      \hspace{22mm}
      \subfigure[AGV ]{\includegraphics[width=60mm]{figs/AGV}}
    \end{center}
    \caption{Robots de Logística}
    \label{fig:logistica}
  \end{figure}


Los AMR ofrecen una mayor flexibilidad, ya que utilizan un sistema de mapeado en tiempo real siendo así posible evitar obstáculos de forma inteligente, mientras que los AGV siguen rutas fijas y predefinidas.

\vspace{0.5cm}

Por último, cabe destacar la aplicación de la robótica de servicio en el ámbito asistencia, un campo que ha experimentado un notable desarrollo en los últimos años debido a la creciente demanda social derivada del envejecimiento de la población y del aumento de personas en situación de dependencia. Un ejemplo de este tipo de robots es ROBERT, desarrollado por la compañía danesa \textit{Life Science Robotics}\footnote{https://www.lifescience-robotics.com/} (Figura \ref{fig:kuka}). 

\vspace{0.2cm}

\begin{figure} [h!]
  \begin{center}
    \includegraphics[width=8cm]{figs/kuka}
  \end{center}
  \caption{\textit{ROBERT®}, de Life Science Robotics}
  \label{fig:kuka}
\end{figure}\


Estos robots están diseñados para apoyar a personas mayores o dependientes en la realización de tareas cotidianas, como recordatorios de medicación, ayuda para la movilidad o la monitorización del estado de salud mediante el seguimiento de sus signos vitales.

\vspace{0.2cm}

El uso de estos sistemas favorece la autonomía de la persona, ayuda al control de su salud y reduce el sentimiento de soledad. 
En muchos casos, estos robots no solo prestan apoyo funcional, sino que también buscan interactuar con las personas de manera más cercana, estableciendo una relación humano-robot. Este tipo de interacción ha dado lugar a una rama específica de la robótica conocida como robótica social.


\vspace{0.5cm}

\subsection{Robótica Social} 
\label{sec:Robótica Social}

La robótica social es un área en expansión que se caracteriza por la interacción directa con las personas.
A diferencia de otros ámbitos de la robótica, centrados principalmente en la automatización de tareas, este tipo de robots como el Pepper de \textit{SoftBank Robotics}\footnote{https://www.softbankrobotics.com/} (Figura \ref{fig:pepper}), requiere un enfoque distinto, ya que su objetivo principal es lograr una interacción fluida y natural entre el sistema y el usuario.

\vspace{0.2cm}

\begin{figure} [h!]
  \begin{center}
    \includegraphics[width=6cm]{figs/pepper}
  \end{center}
  \caption{\textit{Pepper}, de \textit{SoftBank Robotics}}
  \label{fig:pepper}
\end{figure}\


Esta tecnología tiene diversas aplicaciones en la actualidad. Por ejemplo, en el ámbito de la educación, la robótica social permite un aprendizaje más interactivo, favoreciendo la comprensión de conceptos complejos. Ejemplos de este robots se encuentran Bee Bot (Figura \ref{fig:eduacion} a) de la empresa británica TTS Group\footnote{https://www.tts-group.co.uk/programmable-robots/bee-bot/?srsltid=AfmBOoqWRK6mklOyDS1-v2x0iodZU5EapgSx7ERiS1XcRp87WLBetqfy} el cual ayuda a los niños pequeños a aprender conceptos básicos de programación; y OWI 535 (Figura \ref{fig:eduacion} b), creado por OWI Inc\footnote{https://owirobot.com/robot-kits/?srsltid=AfmBOopkaUwORToq60GVbi7tLzf-vIib6PMYAY3umIXSIIugSkrGy2WY} destinado a un público mayor, con el objetivo de enseñar fundamentos básicos de electrónica, mecánica y robótica.
Estos robots ofrecen una experiencia de aprendizaje personalizada, adaptándose a las necesidades y ritmo de cada estudiante. 

\vspace{0.2cm}

  \begin{figure}[H]
    \begin{center}
      \subcapcentertrue
      \subfigure[Bee Bot]{\includegraphics[width=55mm]{figs/bee}}
      \hspace{22mm}
      \subfigure[OWI 535]{\includegraphics[width=55mm]{figs/owi}}
    \end{center}
    \caption{Robots Educativos}
    \label{fig:eduacion}
  \end{figure}



Por otro lado, también pueden desempeñar un papel importante en el ámbito sanitario, especialmente en las intervenciones terapéuticas, como el robot Nuka (Figura \ref{fig:nuka}).
Los robots sociales pueden ayudar a personas con capacidades funcionales reducidas o necesidades especiales a desarrollar habilidades, así como brindar apoyo emocional. \cite{social}


\vspace{0.2cm}

\begin{figure} [h!]
  \begin{center}
    \includegraphics[width=7cm]{figs/nuka}
  \end{center}
  \caption{\textit{Nuka}, de Takanori Shibata}
  \label{fig:nuka}
\end{figure}\


Por último, los robot sociales también se han hecho hueco en el ámbito del entretenimiento, donde se emplean como herramientas lúdicas e interactivas como por ejemplo, Sony Aibo(Figura \ref{fig:entretenimiento}), una mascota robótica diseñada por Sony\footnote{https://us.aibo.com/}.

\vspace{0.2cm}

\begin{figure} [h!]
  \begin{center}
    \includegraphics[width=16cm]{figs/aibo}
  \end{center}
  \caption{\textit{Aibo}, de Sony}
  \label{fig:entretenimiento}
\end{figure}\



Todas estas aplicaciones comparten un elemento en común: la necesidad de una interacción fluida y natural entre humano y robot. Para comprender, diseñar y evaluar estas interacciones, ha surgido un campo de estudio específico denominado Interacción Humano-Robot (HRI).




\vspace{0.5cm}

\section{Interacción Humano-Robot (HRI)}
\label{sec:segundaseccion}

La Interacción Humano-Robot \textit{(Human Robot Interaction, HRI)} se define en \cite{hri} como un campo de estudio dedicado a 
comprender, diseñar y evaluar sistemas robóticos para su uso por o junto con humanos. 

\vspace{0.5cm}

En las primeras etapas, la investigación en HRI se centró en los aspectos relacionados con la movilidad. Sin embargo, con el tiempo, se buscó que el robot tuviera comportamientos más naturales y sociales, con el objetivo de mejorar la interacción entre el robot y la persona \cite{hri2}.


\vspace{0.2cm}

\begin{figure} [h!]
  \begin{center}
    \includegraphics[width=7cm]{figs/kismet}
  \end{center}
  \caption{\textit{Kismet}, de Cynthia Breazeal}
  \label{fig:kismet}
\end{figure}\


En este contexto surge Kismet (Figura \ref{fig:kismet}), un robot desarrollado en el \textit{Massachusetts Institute of Technology (MIT)}\footnote{https://www.mit.edu/} a finales de los años 90. Este sistema marcó un hito en la robótica social al demostrar la capacidad de diseñar robots capaces de percibir, procesar y expresar emociones.

\vspace{0.5cm}

No obstante, a medida que los robots incorporan comportamientos cada vez más expresivos y rasgos similares a la apariencia humana, surgen nuevos retos en la interacción humano-robot.
Aunque estos avances puedan favorecer la aceptación de la tecnología, también pueden provocar rechazo en las personas.
Este fenómeno se conoce como Valle Inquietante (\textit{Uncanny Valley}) (Figura \ref{fig:inqu}), término acuñado por el profesor Masahiro Mori. Esta hipótesis plantea que, a medida que los robots adquieren una apariencia más similar a la humana, la respuesta emocional de las personas tiende a volverse más positiva. Sin embargo, cuando el grado de semejanza es excesivo, puede producirse una reacción negativa, como rechazo o incomodidad.

\vspace{0.2cm}


\begin{figure}[ht!]
	\centering
	\begin{minipage}{0.4\linewidth}
		\centering
		\includegraphics[width=\linewidth]{figs/Uncanny-valley}
		\caption*{\centering Geminoid-DK y Schärfe, de Hiroshi Ishiguro}
	\end{minipage}
	\hspace{2cm}
	\begin{minipage}{0.3\linewidth}
		\centering
		\includegraphics[width=\linewidth]{figs/Uncanny-valley-grafica}
		\caption*{\centering Gráfica Valle Inquietante (Imagen obtenida de Wikipedia) }
	\end{minipage}
	\caption{Uncanny Valley}
	\label{fig:inqu}
\end{figure}

Se han propuesto diversas hipótesis para el fenómeno del Valle Inquietante. Una de las más extendidas tiene origen biológico y evolutivo, según el cual el cerebro humano es capaz de percibir sutiles diferencias en aquello que aparenta ser humano. Aunque el robot presente una apariencia muy similar a la humana, pequeñas diferencias en el movimiento, la expresión facial, o la coordinación pueden activar mecanismos de alerta para protegernos de lo que podría ser una potencial amenaza \cite{hri3}.

\vspace{0.5cm}

Otra teoría, mencionado en el articulo de Martínez, relaciona este fenómeno con la falta de coherencia entre la apariencia antropomórfica del robot y su comportamiento real. Esta discrepancia provoca que el sistema sea percibido como casi humano pero no lo suficiente, provocando falta de empatía e incluso rechazo.
Este efecto resulta especialmente relevante en el diseño de robots sociales, donde la apariencia y el comportamiento juegan un papel fundamental en la percepción del usuario.

\vspace{0.5cm}

Para lograr una interacción humano-robot natural y fluida, el robot debe ser capaz de percibir e interpretar el entorno que lo rodea. En este contexto, la visión artificial desempeña un papel muy importante, ya que permite al robot analizar su entorno, reconocer rostros, expresiones faciales y reaccionar de manera coherente durante la interacción.

\vspace{0.5cm}

\section{Visión Artificial}
\label{sec:Vision Artificial}

La visión artificial es la disciplina encargada de obtener, procesar y analizar información visual proveniente de imágenes y vídeos . Estos procesos pueden realizarse en tiempo real o de manera diferida dependiendo de su aplicación.

\vspace{0.5cm}

Aunque su uso se ha extendido a lo largo de los años, sus orígenes se remontan a la década de 1960, cuando Lawrence Roberts publicó \textit{Percepción mecánica de sólidos tridimensionales} \cite{visionbook}. Este trabajo impulsó una gran cantidad de investigación en el laboratorio de inteligencia artificial del MIT y otras instituciones, donde se analizaban la visión por computadora aplicada al reconocimiento de bloques y objetos simples. 

\vspace{0.5cm}

En la actualidad, la visión artificial cuenta con multitud de aplicaciones en diversos ámbitos, entre los que destacan:

\begin{itemize}
	\item \texttt{Agricultura}. Permite evaluar la maduración de la fruta para la cosecha  (Figura \ref{fig:vision} a) como FFRobot de FFRobotics\footnote{https://www.ffrobotics.com/} y detectar imperfecciones para su selección.
	\item \texttt{Seguridad}. Sistemas de vigilancia y control de tráfico utilizan algoritmos de visión artificial; por ejemplo, para reconocer matrículas o monitorizar vehículos en carretera  (Figura \ref{fig:vision} b).
	\item \texttt{Metrología}. Facilita mediciones precisas y automáticas de piezas y objetos en procesos industriales.
	\item \texttt{Sanidad}. Se emplea para la extracción automática de datos y análisis de imágenes médicas, reduciendo la carga administrativa y mejorando la precisión de diagnósticos.
\end{itemize}

\vspace{0.2cm}

  \begin{figure}[H]
    \begin{center}
      \subcapcentertrue
      %\subfigure[Radar]{\includegraphics[width=55mm]{figs/radar}}
      \subfigure[Radar]{\includegraphics[height=40mm]{figs/radar}}
      \hspace{22mm}
      \subfigure[FFRobot]{\includegraphics[width=60mm]{figs/ffrobot}}
    \end{center}
    \caption{Sistemas con Visión Artificial}
    \label{fig:vision}
  \end{figure}


Para que un robot pueda interactuar con su entorno y con las personas de forma natural, no basta con percibirlos, es necesario que pueda reconocer patrones, identificar rostros y adaptarse al comportamiento del usuario. 
En este contexto, el \textit{Machine Learning} proporciona los métodos y algoritmos que permiten al robot aprender de los datos visuales y tomar decisiones. 

\vspace{0.5cm}

\section{Machine Learning}
\label{sec:segundaseccion}

Hoy en día, las nuevas tecnologías están cada vez más presentes en nuestra sociedad, transformando la forma en que trabajamos, aprendemos e interactuamos.
En esta época de cambios surge la IA como disciplina, dedicada a dotar a la máquina de capacidades asociadas normalmente con los humanos, como el razonamiento, aprendizaje o la capacidad de tomar decisiones.

\vspace{0.5cm}

Una de las ramas más relevantes de la IA es el \textit{Machine Learning}, centrado en algoritmos capaces de identificar patrones y elaborar predicciones o tomar decisiones basadas en ellos. Este enfoque permite que los sistemas aprendan de la experiencia y se adapten a situaciones sin necesidad de programación explícita.

\vspace{0.5cm}

El proceso de aprendizaje automático consta de una serie de etapas (Figura \ref{fig:ML}) fundamentales.

\begin{enumerate}
	\item \texttt{Recolección de datos}. Para obtener un comportamiento adecuado del sistema es importante contar con datos representativos y de calidad. Aunque la cantidad de datos es importante, resulta necesario encontrar un equilibrio entre cantidad y calidad.
	\item \texttt{Preparación de datos}. Los datos pueden contener ruido, valores incompletos o inconsistencias, lo que puede dar lugar a resultados engañosos. Por ello, es necesario preparar minuciosamente los datos antes de su utilización.
	\item \texttt{Elección del modelo}. Existen multitud de modelos de aprendizaje automático, y su elección depende del objetivo del problema. Entre ellos se encuentran algoritmos de clasificación, predicción, regresión lineal, entre otros muchos.
	\item \texttt{Entrenamiento del modelo}. En esta etapa, el modelo aprende a partir de un conjunto de datos de entrenamiento para ajustar los hiperparámetros y aproximarse al resultado esperado.
	\item \texttt{Evaluación del modelo}. Se comparan las predicciones obtenidas con los valores reales mediante métricas específicas, con el objetivo de medir el rendimiento y la precisión del modelo ya entrenado.
	\item \texttt{Ajuste de hiperparámetros (\textit{Parameter Tuning}) }. En función de los resultados obtenidos durante la etapa de evaluación, se ajustan los hiperparámetros del modelo y se repite el proceso de entrenamiento para mejorar el rendimiento.
	\item \texttt{Predicción}. Finalmente, el modelo entrenado se utiliza con nuevos datos, no vistos previamente, para generar predicciones o tomar decisiones, dando paso al final del aprendizaje.
\end{enumerate}

\begin{figure} [h!]
  \begin{center}
    \includegraphics[width=7cm]{figs/ML}
  \end{center}
  \caption{Etapas del aprendizaje automático (Imagen generada por IA)}
  \label{fig:ML}
\end{figure}\

Existen diferentes enfoques dentro del aprendizaje automático, que se clasifican en función del modelo de entrenamiento que usan. A continuación, se describen los principales tipos.

\vspace{0.5cm}

\subsection{Aprendizaje supervisado}
\label{sec:Aprendizaje supervisado}

El aprendizaje supervisado parte de la premisa de asociar datos de entradas con salidas conocidas. 
En este enfoque, el modelo se entrena a partir de conjuntos de datos previamente etiquetados, lo que permite aprender una relación entre las entradas y salidas esperadas.
Entre los algoritmos más utilizados en el aprendizaje supervisado se encuentran la regresión logística, \textit{K-Nearest Neighbors (KNN)} (Figura \ref{fig:supervisado} a) o los árboles de decisión (Figura \ref{fig:supervisado} b).
El uso del aprendizaje supervisado está ampliamente extendido en la actualidad, ya que ofrece buenos resultados en problemas donde se dispone de datos etiquetados.

\vspace{0.2cm}

  \begin{figure}[H]
    \begin{center}
      \subcapcentertrue
      \subfigure[\textit{K-Nearest Neighbors (KNN)}]{\includegraphics[width=60mm]{figs/regresion}}
      \hspace{22mm}
      \subfigure[Árbol de decisiones]{\includegraphics[width=60mm]{figs/tree}}
    \end{center}
    \caption{Algoritmos de aprendizaje supervisado}
    \label{fig:supervisado}
  \end{figure}



El aprendizaje supervisado se utiliza con regularidad en el campo de la visión artificial, concretamente en tareas de reconocimiento y clasificación de imágenes, como la detección y el reconocimiento facial.
Otra aplicación destacable de este tipo de aprendizaje es la conversión de voz a texto, donde los modelos se entrenan a partir de entradas de audio previamente etiquetadas, para reconocer patrones del habla y transcribirlos a lenguaje escrito.

\vspace{0.5cm}

\subsection{Aprendizaje no supervisado}
\label{sec:Aprendizaje no supervisado}

El aprendizaje no supervisado se basa en el uso datos no etiquetados con el objetivo de identificar patrones.
A diferencia del aprendizaje supervisado, no se conoce la salida de antemano, por lo que el algoritmo aprende sin intervención, agrupando y organizando los datos en función de sus características.

\vspace{0.5cm}

Entre los algoritmos más utilizados en el aprendizaje no supervisado se encuentran los métodos de agrupamiento (\textit{clustering}), \textit{K-Means} (Figura \ref{fig:no-supervisado}), o la reducción de dimensionalidad.

\vspace{0.2cm}

\begin{figure} [h!]
  \begin{center}
    \includegraphics[width=15cm]{figs/k-means}
  \end{center}
  \caption{\textit{K-Means}}
  \label{fig:no-supervisado}
\end{figure}\


Las aplicaciones del aprendizaje no supervisado son muy diversas. Un ejemplo común es la segmentación de clientes, utilizada para entender comportamientos  y ofrecer servicios o contenidos personalizados.
Asimismo, este tipo de aprendizaje se emplea en el análisis de secuencias genómicas,  permitiendo identificar patrones genéticos, marcadores comunes o posibles mutaciones dentro de una población.
 
\vspace{0.5cm}

\subsection{Aprendizaje por refuerzo}
\label{sec:Aprendizaje por refuerzo}

El aprendizaje por refuerzo es un enfoque en el que el agente aprende a través de la interacción con el entorno, basándose en un proceso de prueba y error (Figura \ref{fig:refuerzo}). En este método, el agente recibe \textit{feedback} del entorno en forma de recompensas o penalizaciones, en función de las acciones realizadas.
El objetivo es mejor el rendimiento a lo largo del tiempo, aprendiendo qué acciones realizar en las diferentes situaciones para obtener una mayor recompensa.

\vspace{0.2cm}

\begin{figure} [h!]
  \begin{center}
    \includegraphics[width=9cm]{figs/refuerzo2}
  \end{center}
  \caption{Concepto de Aprendizaje por refuerzo (Imagen generada por IA)}
  \label{fig:refuerzo}
\end{figure}\

Para abordar tareas más complejas y de mayor precisión surge el \textit{Deep Learning}, una rama de la IA y del aprendizaje supervisado.

\vspace{0.5cm}

\section{Deeep Learning}
\label{sec:Deeep Learning}

El \textit{Deep Learning} o aprendizaje profundo, es un subcampo de IA que se centra en el diseño y entrenamiento de redes neuronales artificiales con múltiples capas, inspiradas en el funcionamiento del cerebro humano. Esta redes permiten identificar de forma autónoma patrones complejos en grandes volúmenes de datos, lo que ha supuesto un avance significativo frente a métodos tradicionales.

\vspace{0.5cm}

Su uso más extendido se encuentra en aplicaciones de visión artificial, para el reconocimiento de voz y procesamiento de lenguaje natural.

\vspace{0.5cm}

Aunque el aprendizaje profundo ha supuesto un gran avance para los sistemas actuales, también presenta ciertas limitaciones.
La mayoría de equipos emplean la arquitectura Von Neumann (Figura \ref{fig:Von-Neumann}), una estructura basada en en una unidad de procesamiento y una memoria central que almacena tanto los datos como las instrucciones.

\vspace{0.5cm}

Esta arquitectura tiene ventajas, como la posibilidad de compartir memoria entre instrucciones y datos. Sin embargo, al estar la CPU y la memoria separadas, los datos deben viajar a través de un bus. En aplicaciones de aprendizaje profundo, donde la cantidad de datos a procesar aumenta de forma exponencial, este diseño provoca un consumo elevado de energía y un incremento de los costes, tanto económicos como medioambientales, ya que un mayor número de equipos requiere más recursos y agua  para su refrigeración.

\vspace{0.5cm}

Por todo ello, surge la necesidad de buscar sistemas más económicos y sostenibles, capaces de realizar tareas similares pero con un menor impacto en el medio ambiente y la economía. Este enfoque se conoce como robótica de bajo coste.

\vspace{0.2cm}

\begin{figure} [h!]
  \begin{center}
    \includegraphics[width=9cm]{figs/Von}
  \end{center}
  \caption{Diagrama arquitectura Von Neumann}
  \label{fig:Von-Neumann}
\end{figure}\




\section{Robótica de bajo coste}
\label{sec:Robótica de bajo coste}

Como se ha comentado anteriormente, las limitaciones del hardware tradicional y el elevado consumo energético asociado a los algoritmos de IA evidencian la necesidad de desarrollar soluciones más económicas y sostenibles. En este contexto surge la robótica de bajo coste, cuyo objetivo es facilitar el acceso a sistemas robóticos funcionales mediante el uso de plataformas accesibles y de bajo consumo (Figura \ref{fig:bajo-coste}).


\vspace{0.2cm}

\begin{figure} [h!]
  \begin{center}
    \includegraphics[width=6cm]{figs/bajo-coste}
  \end{center}
  \caption{Ejemplo Robótica de Bajo Coste}
  \label{fig:bajo-coste}
\end{figure}\


La existencia de placas computacionales como la Raspberry Pi permite implementar sistemas capaces de realizar tareas complejas, aunque con ciertas limitaciones en cuanto a potencia de cálculo. No obstante, estas plataformas abren la puerta a numerosas aplicaciones en ámbitos como la educación, investigación y la asistencia.


\vspace{1.5cm}

En este proyecto se presenta un sistema que, mediante Visión Artificial, Machine Learning y Deep Learning, es capaz de reconocer personas e interactuar con ellas de forma fluida. Además, incorpora una serie de juegos interactivos con el propósito  de estimular cognitivamente a las personas mayores.


\vspace{0.5cm}

En los siguientes capítulos se presentan el estado del arte, los objetivos de este trabajo, las plataformas de desarrollo utilizadas para el desarrollo del proyecto, así como el diseño y la estructura del sistema. Por último, se expondrán las conclusiones obtenidas tras la investigación y la realización del proyecto.

