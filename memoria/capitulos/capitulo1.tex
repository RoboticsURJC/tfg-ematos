\chapter{Introducción}
\label{cap:capitulo1}
\setcounter{page}{1}

\begin{flushright}
\begin{minipage}[]{10cm}
\emph{ La mayor aventura es la que nos espera. Hoy y mañana aún no se han dicho. Las posibilidades, los cambios son todos vuestros por hacer. El molde de su vida en sus manos está para romper.}\\
\end{minipage}\\

J.R.R. Tolkien, \textit{El Hobbit}\\
\end{flushright}

\vspace{1cm}


La robótica ha ido evolucionando a pasos agigantados en la última mitad del siglo XX; sin embargo, ya apuntaba maneras desde mucho antes. 
En la actulidad, la integración de la inteligenia artificial y el aprendizaje automático está impulsando la robótica hacia nuevas posibilidades.

\vspace{0.5cm}

Los robots han pasado de ser herramientas especializadas con capacidades limitadas a sistemas versátiles, adapatables y con un alto grado de autonomía.
La irrupción de modelos avanzados de IA está potenciando el desarrollo de una auténtica revolución.

\vspace{0.5cm}

En el siguiente capítulo se presenta la definicón de robot,  los distintos tipos de robots existentes y algunas de sus principales aplicaciones en la sociedad con el objetivo comprender las bases de este Trabajo de Fin de Grado.\\

\section{La robótica}
\label{sec:miseccion} % etiqueta para luego referenciar esta sección

La robótica es un campo multidisciplinar que surge de la intersección de la ciencia, la ingeniería y la tecnología. Se dedica al estudio, diseño, construcción e implementación de robots. 

Pero, ¿qué es un robot? 

\vspace{0.5cm}

Un robot es una máquina programable capaz de realizar tareas de forma autónoma o semiautónoma.

\vspace{0.5cm}

Sus orígenes pueden remontarse a las antiguas civilicaciones del Antiguo Egipto, que ya desarrollaban modelos matemáticos muy avanzados y construyeron automatismos muy sofisticados, como el reloj de agua. Siglos más tarde, Leonardo da Vinci diseñó el \textit{Automa Cavaliere}~\ref{fig:Da-Vinci} , considerado uno de los primeros robots humanoides de la historia. 

\vspace{0.2cm}

\begin{figure} [h!]
  \begin{center}
    \includegraphics[width=8cm]{figs/Da-Vinci}
  \end{center}
  \caption{\textit{Automa Cavaliere} de Leonardo Da Vinci}
  \label{fig:Da-Vinci}
\end{figure}\


\vspace{0.5cm}

No obstante, no fue hasta 1921 donde se acuñó por primera vez el concepto de "robot", en la obra de  teatro llamada  “R.U.R: ROBOTS UNIVERSALES ROSSUM”  del dramaturgo checo Karel Capek. Con el paso del tiempo, este término acabaría dando nombre a toda una ingeniería: la robótica.

\vspace{0.5cm}

El siglo XX ha supuesto un antes y después en la robótica: en apenas unas décadas, se pasó  de mecanismos que realizaban tareas repetitivas a sistemas complejos y autónomos. Gracias al desarrollo de la robótica moderna, el escritor de ciencia ficción Isaac Asimov publicó \textit{Yo Robot } obra en la que aparecen las conocidas \textit{leyes de la robótica}, las cuales limitan el funcionamiento y aplicaciones de los robot para evitar que supongan una amenaza para la humanidad.

\vspace{0.5cm}

En el año 1961 la compañía estadounidense  \textit{Unimation} desarrolló el primer robot industrial programable, marcando un hito en la automatozación industrial.
A apartir de ese momento aparecieron nuevos robots en otros sectores, permitiendo automatizar ámbitos y tareas que hasta entonces eran inimaginables.

 




En los textos puedes poner palabras en \textit{cursiva}, para aquellas expresiones en sentido \textit{figurado}, palabras como \textit{robota}, que está fuera del diccionario castellano, o bien para resaltar palabras de una colección: \textit{(a)} es la primera letra del abecedario, \textit{(b)} es la segunda, etc.\\

Al poner las dos líneas del anterior párrafo, este aparecerá separado del anterior. Si no las pongo, los párrafos aparecerán pegados. Sigue el criterio que consideres más oportuno.



\subsection{Robótica industrial}
\label{sec:subseccion}

La robótica industrial es la rama de la ingeniería que se encarga del diseño, desarrollo y fabricación de robots que automatizan tareas repetitivas y/o peligrosas en el ámbito industrial. Estos robots se utilizan en entornos controlados para ejecutar tareas de manera precisa y eficiente, mejorando la productividad de los procesos y reduciendo los errores en la cadena de producción.

\vspace{0.5cm}

El origen de los robots industriales se remonta al desarrollo de \textit{Unimate}~\ref{fig:Unimate}, patentado por George Devol en 1961.

\vspace{0.2cm}

\begin{figure} [h!]
  \begin{center}
    \includegraphics[width=8cm]{figs/Unimate}
  \end{center}
  \caption{\textit{Unimate} de George Devol}
  \label{fig:Unimate}
\end{figure}\

\vspace{0.5cm}

La introducción de \textit{Unimate} impulsó el desarrollo de nuevos robots industriales permitiendo la optimización de procesos.

\vspace{0.5cm}

Hoy en día gracias a las nuevas a las nuevas tecnologías como la inteligencia artificial, el internet de las cosas y elaprendizaje automático se ha conseguido que los robots industriales sean capaces tomar decisiones en tiempo real y actuar en consecuencia, adaptándose a entornos cada vez más complejos.


\subsection{Robótica de Servicio}
\label{sec:subseccion}

En la actualidad, los robot de servicio se han convertido en uno de los tipos de robots más habituales en la sociedad.
Pero, ¿cómo se define un robot de servicio?

\vspace{0.5cm}

Los robots de servicio según la\textit{ Federación internacional de Robótica} se define como\textit{ robot de uso personal o profesional que realiza tareas útiles para humanos o equipos} \cite{ISO8373}.

\vspace{0.5cm}

Las aplicaciones de los robots de servicio son muy amplias y abarcan campos muy diversos como la agricultura, construcción, logística, defensa o la medicina, entre muchas otras.

\vspace{0.5cm}

En el campo de la medicina, destaca el robot Da Vinci~\ref{fig:Robot-Medicina}, un sistema quirúrgico controlado por un cirujano, que permite mejorar las capacidades del profesional durante las intervenciones. Este sistema contribuye a aumentar la precisión, reduciendo así los temblores o movimientos involuntarios que pueda tener el cirujano. Además, el robot viene equipado con múltiples cámaras lo que proporciona una visión más detallada del área a intervenir.


\vspace{0.2cm}

\begin{figure} [h!]
  \begin{center}
    \includegraphics[width=8cm]{figs/Robot-Medicina}
  \end{center}
  \caption{\textit{Robot Da Vinci} de George Devol}
  \label{fig:Robot-Medicina}
\end{figure}\

La incorporación de estas nuevas tecnologías al campo de la medicina suponen una gran ventaja tanto para el profesional sanitario como para el paciente, ya que el uso de este sistema permite realizar intervaciones menos invasivas, lo que conlleva una mejor recuperación, con menor sangrado, cicatrices más reducidas y una disminución de las complicaciones postoperatorias.


\vspace{0.5cm}

Otra de las aplicaciones más extendidas de la robótica de servicio son los robots de limpieza.
Se caracyerizan por la navegación autonoma  gracias a un mapa interno que generan. Existen muchos modelos en el mercado que incluyen más o menos complementos tecnológicos. ~\ref{fig:limpieza}


\begin{figure}[ht!]
	\centering
	\begin{minipage}{0.44\linewidth}
		\centering
		\includegraphics[width=6.5cm]{figs/Roomba-j7}
		\caption{Robot Roomba j7.}			
	\end{minipage}
	\hspace{1cm}
	\begin{minipage}{0.4\linewidth}
		\centering
		\includegraphics[width=6.5cm]{figs/Ecovacs}
		\caption{Robot Ecovacs: WinBot 950} 
		
	\end{minipage}
	\caption{Robots de Limpieza}
	\label{fig:limpieza}
\end{figure}



\subsection{Robótica Social}
\label{sec:subseccion}


\section{Vision Artificial}
\label{sec:segundaseccion}

\section{Machine Learning}
\label{sec:segundaseccion}

\section{Interacción Humano-Robot (HRI)}
\label{sec:segundaseccion}


\section{Deep Learning}
\label{sec:segundaseccion}



No olvides incluir imágenes y referenciarlas, como la Figura \ref{fig:roomba}.


Ni tampoco olvides de poner las URLs como notas al pie. Por ejemplo, si hablo de la Robocup\footnote{\url{http://www.robocup.org}}.



\subsection{Números}
\label{sec:subseccion}



En lugar de tener secciones interminables, como la Sección \ref{sec:miseccion}, divídelas en subsecciones.

Para hablar de números, mételos en el entorno \textit{math} de \LaTeX, por ejemplo, $1.5Kg$. También puedes usar el símbolo del Euro como aquí: 1.500\euro.

\subsection{Listas}

Cuando describas una colección, usa \texttt{itemize} para ítems o \texttt{enumerate} para enumerados. Por ejemplo:

\begin{itemize}
 \item \textit{Entorno de simulación.} Hemos usado dos entornos de simulación: uno en 3D y otro en 2D.
 \item \textit{Entornos reales.} Dentro del campus, hemos realizado experimentos en Biblioteca y en el edificio de Gestión.
\end{itemize}\

\begin{enumerate}
 \item Primer elemento de la colección.
 \item Segundo elemento de la colección.
\end{enumerate}\

\paragraph{Referencias bibliográficas}
\label{sec:referencias}

Cita, sobre todo en este capítulo, referencias bibliográficas que respalden tu argumento. Para citarlas basta con poner la instrucción \verb|\cite| con el identificador de la cita. Por ejemplo: libros como \cite{vega12e}, artículos como \cite{vega19b}, URLs como \cite{vega19a}, tesis como \cite{vega18b}, congresos como \cite{vega18a}, u otros trabajos fin de grado como \cite{vega08b}.

Las referencias, con todo su contenido, están recogidas en el fichero \texttt{bibliografia.bib}. El contenido de estas referencias está en formato \texttt{BibTex}. Este formato se puede obtener en muchas ocasiones directamente, desde plataformas como \texttt{Google Scholar} u otros repositorios de recursos científicos.

Existen numerosos estilos para reflejar una referencia bibliográfica. El estilo establecido por defecto en este documento es APA, que es uno de los estilos más comunes, pero lo puedes modificar en el archivo \texttt{memoria.tex}; concretamente, cambiando el campo \verb|apalike| a otro en la instrucción \verb|\bibliographystyle{apalike}|. 

\

\

\

Y, para terminar este capítulo, resume brevemente qué vas a contar en los siguientes.
