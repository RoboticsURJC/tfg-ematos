\chapter{Introducción}
\label{cap:capitulo1}
\setcounter{page}{1}

\begin{flushright}
\begin{minipage}[]{10cm}
\emph{ La mayor aventura es la que nos espera. Hoy y mañana aún no se han dicho. Las posibilidades, los cambios son todos vuestros por hacer. El molde de su vida en sus manos está para romper.}\\
\end{minipage}\\

J.R.R. Tolkien, \textit{El Hobbit}\\
\end{flushright}

\vspace{1cm}


La robótica ha ido evolucionando a pasos agigantados en la última mitad del siglo XX; sin embargo, ya apuntaba maneras desde mucho antes. 
En la actulidad, la integración de la inteligenia artificial y el aprendizaje automático está impulsando la robótica hacia nuevas posibilidades.  

\vspace{0.5cm}

Los robots han pasado de ser herramientas especializadas con capacidades limitadas a sistemas versátiles, adapatables y con un alto grado de autonomía.
La irrupción de modelos avanzados de IA está potenciando el desarrollo de una auténtica revolución.

\vspace{0.5cm}

En el siguiente capítulo se presenta la definicón de robot,  los distintos tipos de robots existentes y algunas de sus principales aplicaciones en la sociedad con el objetivo comprender las bases de este Trabajo de Fin de Grado.\\

\section{La robótica}
\label{sec:miseccion} % etiqueta para luego referenciar esta sección

La robótica es un campo multidisciplinar que surge de la intersección de la ciencia, la ingeniería y la tecnología. Se dedica al estudio, diseño, construcción e implementación de robots. 

Pero, ¿qué es un robot? 

\vspace{0.2cm}

Un robot es una máquina programable capaz de realizar tareas de forma autónoma o semiautónoma.

\vspace{0.5cm}

Sus orígenes pueden remontarse a las antiguas civilicaciones del Antiguo Egipto, que ya desarrollaban modelos matemáticos muy avanzados y construyeron automatismos muy sofisticados, como el reloj de agua. Siglos más tarde, Leonardo da Vinci diseñó el \textit{Automa Cavaliere}~\ref{fig:Da-Vinci} , considerado uno de los primeros robots humanoides de la historia. 

\vspace{0.2cm}

\begin{figure} [h!]
  \begin{center}
    \includegraphics[width=7cm]{figs/Da-Vinci}
  \end{center}
  \caption{\textit{Automa Cavaliere} de Leonardo Da Vinci}
  \label{fig:Da-Vinci}
\end{figure}\


No obstante, no fue hasta 1921 donde se acuñó por primera vez el concepto de robot, en la obra de  teatro llamada  “R.U.R: ROBOTS UNIVERSALES ROSSUM”  del dramaturgo checo Karel Capek. Con el paso del tiempo, este término acabaría dando nombre a toda una ingeniería: la robótica.  \cite{iniciorob}

\vspace{0.5cm}

El siglo XX ha supuesto un antes y después en la robótica: en apenas unas décadas, se pasó  de mecanismos que realizaban tareas repetitivas a sistemas complejos y autónomos. Gracias al desarrollo de la robótica moderna, el escritor de ciencia ficción Isaac Asimov publicó \textit{Yo Robot} obra en la que aparecen las conocidas \textit{leyes de la robótica}, las cuales limitan el funcionamiento y aplicaciones de los robot para evitar que supongan una amenaza para la humanidad.

\vspace{0.5cm}

En el año 1961 la compañía estadounidense  \textit{Unimation} desarrolló el primer robot industrial programable, marcando un hito en la automatozación industrial.
A apartir de ese momento aparecieron nuevos robots en otros sectores, permitiendo automatizar ámbitos y tareas que hasta entonces eran inimaginables.

 







\subsection{Robótica industrial}
\label{sec:subseccion}

La robótica industrial es la rama de la ingeniería que se encarga del diseño, desarrollo y fabricación de robots que automatizan tareas repetitivas y/o peligrosas en el ámbito industrial. Estos robots se utilizan en entornos controlados para ejecutar tareas de manera precisa y eficiente, mejorando la productividad de los procesos y reduciendo los errores en la cadena de producción.

\vspace{0.5cm}

El origen de los robots industriales se remonta al desarrollo de \textit{Unimate}~\ref{fig:Unimate}, patentado por George Devol en 1961.

\vspace{0.2cm}

\begin{figure} [h!]
  \begin{center}
    \includegraphics[width=8cm]{figs/Unimate}
  \end{center}
  \caption{\textit{Unimate} de George Devol}
  \label{fig:Unimate}
\end{figure}\


La introducción de \textit{Unimate} impulsó el desarrollo de nuevos robots industriales permitiendo la optimización de procesos.
Además, favoreció la expansión del uso de la robótica en otros sectores, como la robótica de servicio.

\vspace{0.5cm}



\subsection{Robótica de Servicio}
\label{sec:subseccion}

En la actualidad, los robots de servicio se han convertido en uno de los tipos de robots más habituales en la sociedad.
Pero, ¿cómo se define un robot de servicio?

\vspace{0.5cm}

Según la\textit{ Federación internacional de Robótica} un robot de servicio se define como\textit{ robot de uso personal o profesional que realiza tareas útiles para humanos o equipos} \cite{ISO8373}.

\vspace{0.5cm}

Las aplicaciones de los robots de servicio son muy amplias y abarcan campos muy diversos como la agricultura, la construcción, la logística, la defensa o la medicina, entre muchos otros.

\vspace{0.5cm}

En el ámbito de la medicina destaca el robot Da Vinci~\ref{fig:Robot-Medicina}, un sistema quirúrgico controlado por un cirujano, que permite mejorar las capacidades del profesional durante las intervenciones. Este sistema contribuye a aumentar la precisión, reduciendo así los temblores o movimientos involuntarios que pueda tener el cirujano. Además, el robot viene equipado con múltiples cámaras que proporcionan una visión más detallada del área a intervenir.


\vspace{0.2cm}

\begin{figure} [h!]
  \begin{center}
    \includegraphics[width=8cm]{figs/Robot-Medicina}
  \end{center}
  \caption{\textit{Robot Da Vinci} de George Devol}
  \label{fig:Robot-Medicina}
\end{figure}\

La incorporación de estas nuevas tecnologías en el campo de la medicina suponen una gran ventaja tanto para el profesional sanitario como para el paciente, ya que el uso de este sistema permite realizar intervaciones menos invasivas, favoreciendo una mejor recuperación, con menor sangrado, cicatrices más reducidas y una disminución de las complicaciones postoperatorias.


\vspace{0.5cm}

Otra de las aplicaciones más extendidas de la robótica de servicio son los robots de limpieza. Estos robots se caracterizan por su capacidad de navegación autónoma, basada en la generación de mapas internos que les permiten desplazarse de forma eficiente y evitar obstáculos. 

\vspace{0.2cm}

Existe una amplia variedad de modelos en el mercado, que incorporan distintos niveles de tecnología y funcionalidades. ~\ref{fig:limpieza}


\begin{figure}[ht!]
	\centering
	\begin{minipage}{0.3\linewidth}
		\centering
		\includegraphics[width=\linewidth]{figs/Roomba-j7}
		\caption*{\centering Roomba-j7 }
	\end{minipage}
	\hspace{3cm}
	\begin{minipage}{0.3\linewidth}
		\centering
		\includegraphics[width=\linewidth]{figs/Ecovacs}
		\caption*{\centering Ecovacs Deebot OZMO N8 PRO+}
	\end{minipage}
	\caption{Robots de limpieza}
	\label{fig:limpieza}
\end{figure}

\vspace{0.5cm}

Dentro de la robótica de servicio, la logística constituye uno de los ámbitos en los que estos sistemas han experimentado un mayor desarrollo.
Los robots logísticos se emplean en tareas como el transporte de mercancias dentro de almacenes, optimizanzo los procesos, reduciendo los costes logísticos y aumentando la productividad.

\vspace{0.5cm}

Existen principalmente dos tipos de robots logísticos: los robots móviles autónomos (AMR) y los vehículo de guiado automático (AGV).~\ref{fig:logistica}. 

\vspace{0.2cm}

\begin{figure}[ht!]
	\centering
	\begin{minipage}{0.4\linewidth}
		\centering
		\includegraphics[width=\linewidth]{figs/AMR}
		\caption*{\centering AMR }
	\end{minipage}
	\hspace{2cm}
	\begin{minipage}{0.4\linewidth}
		\centering
		\includegraphics[width=\linewidth]{figs/AGV}
		\caption*{\centering AGV }
	\end{minipage}
	\caption{Robots Logistíca}
	\label{fig:logistica}
\end{figure}


\vspace{0.2cm}

Los AMR ofrecen una mayor flexibilidad, ya que utilizan un sistema de mapeado en tiempo real siendo así posible evitar obstáculos de forma inteligente, mientras que los AGV siguen rutas fijas y predefinidas.

\vspace{0.5cm}

Por último, cabe destacar la aplicación de la robótica de servicio en el ámbito asistencial.

\vspace{0.2cm}

Se trata de un campo que ha experimentado un notable desarrollo en los últimos años, debido a la creciente demanda social derivada del envejecimiento de la población y del aumento de personas en situación de dependencia.~\ref{fig:kuka}. 

\vspace{0.2cm}

\begin{figure} [h!]
  \begin{center}
    \includegraphics[width=8cm]{figs/kuka}
  \end{center}
  \caption{\textit{ROBERT®} de Life Science Robotics}
  \label{fig:kuka}
\end{figure}\


Estos robots están diseñados para apoyar a personas mayores o dependientes en la realización de tareas cotidianas, como recordatorios de medicación, ayuda para la movilidad o la monitorización del estado de salud mediante el seguimiento de sus signos vitales.

\vspace{0.2cm}

El uso de estos sistemas favorece la autonomía de la persona, ayuda al control de su salud y reduce el sentimiento de soledad. 

\vspace{0.2cm}

En muchos casos, estos robots no solo prestan apoyo funcional, sino que también interactuan con las personas, estableciendo una relación humano-robot, lo que da origen a la rama de la robótica conocida como: robótica social




\subsection{Robótica Social}
\label{sec:subseccion}

La robótica social es un área en expansión que se caracteriza por la interacción directa con las personas.
A diferencia de otros ámbitos de la robótica, centrados principalmente en la automatización de tareas, este tipo de robots requiere un enfoque distinto ya que su objetivo principal lograr una interacción fluida y natural entre el sistema y el usuario.~\ref{fig:pepper}.

\vspace{0.2cm}

\begin{figure} [h!]
  \begin{center}
    \includegraphics[width=8cm]{figs/pepper}
  \end{center}
  \caption{\textit{Pepper} de SoftBank Corp}
  \label{fig:pepper}
\end{figure}\


Esta tecnología tiene diversas aplicaciones en la actualidad.

\vspace{0.2cm}

En el ámbito de la eduación, la robótica social permite un apredizaje más interactivo, favoreciendo la comprensión de conceptos complejos.~\ref{fig:eduacion}.

\vspace{0.2cm}

\begin{figure}[ht!]
	\centering
	\begin{minipage}{0.3\linewidth}
		\centering
		\includegraphics[width=\linewidth]{figs/owi}
		\caption*{\centering OWI 535}
	\end{minipage}
	\hspace{1cm}
	\begin{minipage}{0.3\linewidth}
		\centering
		\includegraphics[width=\linewidth]{figs/bee}
		\caption*{\centering Bee Bot }
	\end{minipage}
	\caption{Robots Eduactivos}
	\label{fig:eduacion}
\end{figure}


Estos robots ofrecen una experiencia de aprendizaje personalizada, adaptandose a las necesesidades y ritmo de cada estudiante. 

\vspace{0.5cm}

Por otro lado, también pueden desempeñar un papel importante en el ámbito sanitario, especialmente en las intervenciones terapéuticas~\ref{fig:nuka}.
Los robots sociales pueden ayudar a personas con capacidades funcionales reducidas o necesidades especiales a desarrollar habilidades, así como brindar apoyo emocional. \cite{social}


\vspace{0.2cm}

\begin{figure} [h!]
  \begin{center}
    \includegraphics[width=7cm]{figs/nuka}
  \end{center}
  \caption{\textit{Nuka} deTakanori Shibata}
  \label{fig:nuka}
\end{figure}\


\vspace{0.5cm}

Por último, los robot sociales también se han hecho hueco en el ambito del entretenimiento, donde se emplean como herramientas lúdicas e interactivas. 

\vspace{0.5cm}

Todas estas aplicaciones tienen un factor común: la necesidad de una interacción fluida y natural entre humano y robot. Este factor ha dado lugar a un estudio conocido como Interacción Humano-Robot (HRI).






\section{Interacción Humano-Robot (HRI)}
\label{sec:segundaseccion}

La Interacción Humano-Robot\textit{(Human Robot Interaction, HRI)} fue definida por Michael A. Goodrich and Alan C. Schultz como un campo de estudio dedicado a 
comprender, diseñar y evaluar sistemas robóticos para su uso por o junto con humanos. \cite{hri}

\vspace{0.5cm}

El desarrollo de la HRI se vio favorecida por la aparación de las arquitecturas híbridas, las cuales combinan comportamientos reactivos, que proveen capacidades fundamentales a los robots, con niveles más alto de razonamiento cognitivo. Esta combinación permite una interacción más compleja y natural con los humanos.~\ref{fig:kismet}


\vspace{0.2cm}

\begin{figure} [h!]
  \begin{center}
    \includegraphics[width=7cm]{figs/kismet}
  \end{center}
  \caption{\textit{Kismet} de Cynthia Breazeal}
  \label{fig:kismet}
\end{figure}\


En las primeras etapas, la investigación en HRI se centró en los aspectos relacionados con la movilidad. Sin embargo, con el tiempo, se buscó que el robot tuviera comportamientos más natuarales y sociales, con el objetivo de mejorar la interacción entre el robot y la persona. \cite{hri2}

\vspace{0.5cm}

En este contexto surge Kismet~\ref{fig:kismet}, un robot desarrollado en el Instituto de Massachusetts de Tecnología (MIT) a finales de los años 90. Este sistema marcó un hito en la robótica social al demostrar la capacidad de diseñar robots capaces de percibir, procesar y expresar emociones.

\vspace{0.5cm}

No obstante, a medida que los robots incorporan comportamientos cada más expresivos y rasgos similares  a la apariencia y aspecto humano, surgen nuevos retos en la interacción humano-robot.
Aunque estos avances puedan favorecer la aceptación del sistema también pueden provocar rechazo en las personas.

\vspace{0.5cm}

Este fenómeno se conoce como Valle Inquitante(\textit{Uncanny Valley})~\ref{fig:inqu}, término acuñado por el profesor Masahiro Mori. Esta hipotesis plantea que, a medida que los robots adquieren una apariencia cada vez más similar a la humana, la respuesta emocional de las personas hacia ellos tiende a ser más positiva. Sin embargo, cuando el grado de semejanza es excesivo, se produce una reacción negativa como el rechazo o la incomodidad. 

\vspace{0.2cm}

\begin{figure}[ht!]
	\centering
	\begin{minipage}{0.4\linewidth}
		\centering
		\includegraphics[width=\linewidth]{figs/Uncanny-valley}
		\caption*{\centering Geminoid F de Hiroshi Ishiguro}
	\end{minipage}
	\hspace{2cm}
	\begin{minipage}{0.4\linewidth}
		\centering
		\includegraphics[width=\linewidth]{figs/Uncanny-valley-grafica}
		\caption*{\centering Foto sacada de Wikipedia }
	\end{minipage}
	\caption{Uncanny Valley}
	\label{fig:inqu}
\end{figure}

Se han propuesto diversas hipótesis para el fenómeno del Valle Inquietante. Una de las más extendidas tiene origen biológico y evolutivo, según el cual el cerebro humano es capaz de percibir sutiles diferencias en aquello que aparenta ser humano. Aunque el robot presente una apariencia muy similar a la humana, pequeñas diferencias en el movimiento, la expresión facial, o la coordinación pueden activar mecanismos de alerta para protegernos de lo que podría ser una potencial amenaza.\cite{hri3}

\vspace{0.5cm}

Otra teoría relaciona este fenómeno con la  falta de coherencia entre la apariencia antropomórfica del robot y su comportamiento real. Esta discrepancia provoca que el sistema sea percibido como casi humano pero no lo suficiente, provocando falta de empatía e incluso rechazo.

\vspace{0.5cm}

Este efecto resulta especialmente relevante en el diseño de robots sociales, donde la apariencia y el comportamiento juegan un papel fundamental en la percepción del usuario.

\vspace{0.5cm}

Para lograr una interacción humano-robot natural y fluida, el robot debe de ser capaz de percibir e interpretar el entorno que lo rodea. En este contexto, la visión artificial desempeña un papel muy importante, ya que permite al robot analizar su entorno, reconocer rostros, expresiones faciales y reaccionar de manera coherente durante la interacción.



\section{Vision Artificial}
\label{sec:segundaseccion}

La visión artificial es la disciplina encargada de obtener, procesar, analizar información visual provemiente de imágenes y vídeos . Estos procesos pueden realizarse en tiempo real o de manera diferida dependiendo de su aplicación.

\vspace{0.5cm}

Aunque su uso se ha extendido a lo largo de los años, sus orígenes se remontan a la década de 1960, cuando Lawrence Roberts publicó \textit{Percepción mecánica de sólidos tridimensionales}. Este trabajo implulsó una gran cantidad de investigación en el laboratorio de inteligencia artificial del MIT y otras instituciones, donde se analizaban la visión por computadora aplicada al reconocimiento de bloques y objetos simples. 

\vspace{0.2cm}

En la actulidad, la visión artificial cuenta con multitud de aplicaciones en diversos ámbitos, entre los que destancan ~\ref{fig:vision}:

\begin{itemize}
	\item \textit{Agricultura}. Permite evaluar la maduración de la fruta para la cosecha y detectar imperfecciones para su selección.
	\item \textit{Seguridad}. Sistemas de vigilancia y control de tráfico utilizan algoritmos de visión artificial, por ejemplo para reconocer matrículas o monitorizar vehículos en carretera.
	\item \textit{Metrología}. Facilita mediciones precisas y automáticas de piezas y objetos en procesos industriales.
	\item \textit{Sanidad}. Se emplea para la extracción automática de datos y análisis de imágenes médicas, reduciendo la carga administrativa y mejorando la precisión de diagnósticos.
\end{itemize}


\vspace{0.2cm}

\begin{figure}[ht!]
	\centering
	\begin{minipage}{0.4\linewidth}
		\centering
		\includegraphics[width=\linewidth]{figs/radar}
		\caption*{\centering Radar}
	\end{minipage}
	\hspace{2cm}
	\begin{minipage}{0.4\linewidth}
		\centering
		\includegraphics[width=\linewidth]{figs/robot-apple}
		\caption*{\centering \textit{Robot Apple Harvester} }
	\end{minipage}
	\caption{Sistemas con Visión Artificial}
	\label{fig:vision}
\end{figure}

Para que un robot pueda interactuar con su entorno y con las personas de forma natural, no basta con percibirlos, es necesario que pueda reconocer patrones, identificar rostros y adaptarse al comportamiento del usuario. 

\vspace{0.2cm}

En este contexto, el \textit{Machine Learning} proporciona los métodos y algoritmos que permiten al robot aprender de los datos visuales y tomar decisiones. 



\section{Machine Learning}
\label{sec:segundaseccion}

En los tiempos que corren, las nuevas tecnologías están cada vez más presentes en nuestra sociedad, transformando la forma en que trabajamos, aprendemos e interactuamos.
En esta época de cambios surge la Inteligencia Artificicial (IA) como disciplina, dedicada a dotar a la maquina de capacidades asociadas normalmente con los humanos como el razonamiento, aprendizaje o la capacidade de tomar decisiones.

\vspace{0.5cm}

Una de las ramas más relevantes de la IA es el \textit{Machine Learning}, centrado en algoritmos capaces de identificar patrones y elaborar predicciones o tomar decisiones basadas en ellos. Este enfoce permite que los sistemas aprendan de la experiencia y se adapten a situaciones sin necesidad de programación explícita.

\vspace{0.5cm}

El proceso de aprendizaje automático consta de una serie de etapas fundamentales~\ref{fig:ML}.

\begin{itemize}
	\item \textit{Recolección de datos}. Para obtener un comportamiento adecuado del sistema es importante contar con datos representativos y de calidad. Aunque la cantidad de datos es importante, resulta necesario encontrar un equilibrio entre cantidad y calidad.
	\item \textit{Preparación de datos}. Los datos pueden contener ruido, valores incompletos o inconsistencias, lo que puede dar lugar a resultados engañosos. Por ello, es necesario preparar minuciosamente los datos antes de su utilización.
	\item \textit{Elección del modelo}. Existen multitud de modelos de aprendizaje automático, y su elección depende del objetivodel problema. Entre ellos se encuentran algoritmos de clasificación, predicción, regresión lineal, entre otros muchos.
	\item \textit{Entrenamiento del modelo}. En esta etapa, el modelo aprende a partir de un conjunto de datos de entrenamiento para ajustar los hiperparámetros y aproximarse al resultado esperado.
	\item \textit{Evaluación del modelo}. Se comparan las predicciones obtenidas con los valores reales mediante métricas específicas, con el objetivo de medir el rendimiento y la precisión del modelo ya entrenado.
	\item \textit{Ajuste de hiperparámetros (\textit{Parameter Tuning}) }. En función de los resultados obtenidos durante la etapa de evalución, se ajustan los hiperparámetros del modelo y se repite el proceso de entrenamiento para mejorar el rendimiento.
	\item \textit{Predicción}. Finalmente, el modelo entrenado se utiliza con nuevos datos, no vistos previamente, para generar predicciones o tomar decisiones, dando paso al final del aprendizaje.
\end{itemize}

\begin{figure} [h!]
  \begin{center}
    \includegraphics[width=5.5cm]{figs/ML}
  \end{center}
  \caption{Imagen generada por Inteligencia Artificial}
  \label{fig:ML}
\end{figure}\

Existen diferentes enfoques dentro del aprendizaje autómatico, que se clasifican en función del modelo de entrenamiento que usan.


\subsection{Aprendizaje supervisado}
\label{sec:subseccion}

El aprendizaje supervisado parte de la premisa de asociar datos de entradas con salidas conocidas. 
En este enfoque, el modelo se entrena a partir de conjuntos de datos previamente etiquetados, lo que permite aprender una relación entre las entradas y salidas esperadas.

\vspace{0.5cm}

Entre los algoritmos más utilizados en el aprendizaje supervisado se encuentran la regresión logística, \textit{K-Nearest Neighbors (KNN)} o los árboles de decisión.~\ref{fig:supervisado}

\vspace{0.2cm}

\begin{figure}[ht!]
	\centering
	\begin{minipage}{0.4\linewidth}
		\centering
		\includegraphics[width=\linewidth]{figs/regresion}
		\caption*{\centering Regresión Logística}
	\end{minipage}
	\hspace{2cm}
	\begin{minipage}{0.4\linewidth}
		\centering
		\includegraphics[width=\linewidth]{figs/tree}
		\caption*{Árbol de decisiones}
	\end{minipage}
	\caption{Algoritmos de aprendizaje supervisado}
	\label{fig:supervisado}
\end{figure}


\vspace{0.5cm}

El uso del aprendizaje supervisado está ampliamente extendido en la actualidad, ya que ofrece buenos resultadosen problemas donde se dispone de datos etiquetados.

\vspace{0.5cm}

El aprendizaje supervisado es usado con regularidad en el campo de la visión artificial, concretamente en tareas de reconocimiento y clasificiación de imágenes, como la detección y el reconocimiento facial.

\vspace{0.5cm}

Otra aplicación a resaltar de este tipo de aprendizaje es la conversión de voz a texto, donde los modelos se entrenan a partir de entradas de audio previamente etiquetadas para reconocer patrones del habla y trasnscribirlos a lenguaje escrito.



\subsection{Aprendizaje no supervisado}
\label{sec:subseccion}

El aprendizaje no supervisado se basa en el uso datos no etiquetados con el objetivo de identificar patrones.
A diferencia del aprendizaje supervisado, no se conoce la salida de antemano, por lo que el algoritmo aprende sin intervención, agrupando y organizando los datos en función de sus características.

\vspace{0.5cm}

Entre los algoritmos más utilizados en el aprendizaje no supervisado se encuentran los métodos de agrupamiento (\textit{clustering}), \textit{K-Means}, o la reducción de dimensionalidad.~\ref{fig:no-supervisado}

\vspace{0.2cm}

\begin{figure}[ht!]
	\centering
	\begin{minipage}{0.4\linewidth}
		\centering
		\includegraphics[width=\linewidth]{figs/cluster}
		\caption*{\centering Agrupamiento}
	\end{minipage}
	\hspace{2cm}
	\begin{minipage}{0.4\linewidth}
		\centering
		\includegraphics[width=\linewidth]{figs/reduccion}
		\caption*{Reducción de dimensionalidad}
	\end{minipage}
	\caption{Algoritmos de aprendizaje no supervisado}
	\label{fig:no-supervisado}
\end{figure}


Las apliaciones del aprendizaje no supervisado son muy diversas. Un ejemplo omún es la segmentación de clientes, utilizada para entender comportamientos  y ofrecer servicios o contenidos personalizados.

\vspace{0.5cm}

Asimismo, este tipo de aprendizaje se emplea en el análisis de secuencias genómicas,  permitiendo identifiicar patrones genéticos, marcadores comunes o posibles mutaciones dentro de una población.
 

\subsection{Aprendizaje por refuerzo}
\label{sec:subseccion}

El aprendizaje por refuerzo es un enfoque en el que el sistema aprende a través de la interacción con el entorno, basandose en un proceso de prueba y error. En este método, el agente recibe \textit{feedback} del entorno en forma de recompensas o penalizaciones, en función de las acciones realizadas.

\vspace{0.5cm}

El obetivo es mejor el rendimiento a lo largo del tiempo, aprendiendo qué acciones realizar en las diferentes situaciones para obtener una mayor recompensa.~\ref{fig:refuerzo}

\vspace{0.2cm}

\begin{figure} [h!]
  \begin{center}
    \includegraphics[width=7cm]{figs/refuerzo}
  \end{center}
  \caption{Concepto de Aprendizaje por refuerzo}
  \label{fig:refuerzo}
\end{figure}\



\section{Deeep Learning}
\label{sec:segundaseccion}


\section{Robótica de bajo coste}
\label{sec:segundaseccion}






Ni tampoco olvides de poner las URLs como notas al pie. Por ejemplo, si hablo de la Robocup\footnote{\url{http://www.robocup.org}}.



\subsection{Números}
\label{sec:subseccion}



En lugar de tener secciones interminables, como la Sección \ref{sec:miseccion}, divídelas en subsecciones.

Para hablar de números, mételos en el entorno \textit{math} de \LaTeX, por ejemplo, $1.5Kg$. También puedes usar el símbolo del Euro como aquí: 1.500\euro.

\subsection{Listas}

Cuando describas una colección, usa \texttt{itemize} para ítems o \texttt{enumerate} para enumerados. Por ejemplo:


\begin{enumerate}
 \item Primer elemento de la colección.
 \item Segundo elemento de la colección.
\end{enumerate}\



\

\

\

Y, para terminar este capítulo, resume brevemente qué vas a contar en los siguientes.
